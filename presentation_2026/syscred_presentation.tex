\documentclass{beamer}

% --- Thème et Configuration ---
\usetheme{Warsaw}
\usecolortheme{default}
% Désactiver les ombres qui créent des boîtes noires sur certains visualiseurs
\setbeamertemplate{blocks}[rounded][shadow=false] 
\setbeamertemplate{title page}[default][colsep=-4bp,rounded=true,shadow=false]
\usepackage[utf8]{inputenc}
\usepackage[T1]{fontenc}
\usepackage[french]{babel}
\usepackage{graphicx}
\usepackage{booktabs}
\usepackage{amsmath,amssymb}
\usepackage{natbib} % Pour \citet et \citep

% Configuration de la bibliographie pour ressembler au style APA
\bibliographystyle{apalike}

% --- Métadonnées ---
\title[sysCRED : Architecture Hybride Neuro-Symbolique]{Proposition de Recherche Doctorale (DIC-9411) :\\Architecture, Formalisation et Implantation du Système sysCRED}
\subtitle{Une Approche Hybride Neuro-Symbolique pour la Crédibilité et le Raisonnement en Informatique Cognitive}
\author{Dominique Loyer}
\institute{Université du Québec à Montréal (UQAM)\\Doctorat en informatique cognitive}
\date{28 janvier 2026}

\begin{document}

% --- Titre ---
\begin{frame}
  \titlepage
\end{frame}

% --- Sommaire ---
\begin{frame}{Plan de la présentation}
  \tableofcontents
\end{frame}

% --- Section 1: Introduction ---
\section{Introduction et Problématique}

\begin{frame}{Introduction : La 3ème Vague de l'IA}
  \begin{itemize}
    \item \textbf{Oscillation historique} : Symbolisme (Règles) $\leftrightarrow$ Connexionnisme (Réseaux de neurones).
    \item \textbf{Convergence actuelle} : Nécessité de systèmes hybrides (Neuro-Symbolique).
    \item Allier la \textit{généralisation neuronale} à la \textit{rigueur symbolique} \citep{Hitzler2025Survey}.
  \end{itemize}
\end{frame}

\begin{frame}{Le « Léviathan Algorithmique »}
  \begin{itemize}
    \item \textbf{Contexte} : Gouvernance automatisée par des algorithmes opaques \citep{Hakim2025Cybersecurity}.
    \item \textbf{Bureaucratie numérique} : Vecteurs latents inintelligibles vs Bureaucratie traditionnelle (règles écrites).
    \item \textbf{Déficit de crédibilité} des LLM (Large Language Models) :
      \begin{itemize}
        \item Moteurs de corrélation statistique, pas de modèles causaux.
        \item Hallucinations factuelles et grande assurance trompeuse.
      \end{itemize}
  \end{itemize}
\end{frame}

\begin{frame}{Véracité vs Crédibilité}
  \begin{block}{Distinction Épistémologique}
    \begin{enumerate}
      \item \textbf{Véracité (Truthfulness)} : Correspondance énoncé/fait observable.
      \item \textbf{Crédibilité (Credibility)} : Méta-propriété (fiabilité source, processus, cohérence) \citep{Pan2025LLMKG}.
    \end{enumerate}
  \end{block}
  \begin{alertblock}{Problème}
    Les LLM compressent les sources et perdent le contexte. Les systèmes symboliques purs (GOFAI) sont fragiles face au web.
  \end{alertblock}
\end{frame}

% --- Section 2: Hypothèse et Objectifs ---
\section{Hypothèse et Objectifs}

\begin{frame}{Hypothèse de Recherche}
  \begin{center}
    \Large
    Seule une \textbf{architecture hybride neuro-symbolique}, intégrant une ontologie de la crédibilité (Système 2) sur un modèle de langage perceptif (Système 1), permet d'atteindre la fiabilité requise.
  \end{center}
\end{frame}

\begin{frame}{Objectifs Spécifiques (d'ici avril 2026)}
  \begin{itemize}
    \item \textbf{Théorique (Modélisation)} :
      \begin{itemize}
        \item Formaliser une \textit{Credibility Ontology} (biais, conflit d'intérêt, preuve, expertise).
        \item Dépasser le binaire Vrai/Faux.
      \end{itemize}
    \item \textbf{Technique (Implémentation)} :
      \begin{itemize}
        \item Concevoir \textbf{sysCRED} (System for Credibility and Reasoning utilizing Expert Dynamics).
        \item Extraction neuro-symbolique et peuplement dynamique de graphe (GraphRAG).
      \end{itemize}
    \item \textbf{Méthodologique (Validation)} :
      \begin{itemize}
        \item Double métrique : Précision (ML) + Qualité d'explication (Cognitif).
      \end{itemize}
  \end{itemize}
\end{frame}

% --- Section 3: Fondements Théoriques ---
\section{Fondements Théoriques}

\begin{frame}{Système 1 et Système 2}
  Basé sur la \textit{Dual Process Theory} (Kahneman) adaptée à l'IA \citep{Yang2025Reasoning}.

  \begin{columns}[T]
    \begin{column}{0.48\textwidth}
      \textbf{Système 1 (Intuitif)}
      \begin{itemize}
        \item Rapide, parallèle, associatif.
        \item Réseaux de neurones profonds.
        \item Perception (Vision, NLP).
        \item $\rightarrow$ \textit{Neural Interpreter}
      \end{itemize}
    \end{column}
    \begin{column}{0.48\textwidth}
      \textbf{Système 2 (Délibératif)}
      \begin{itemize}
        \item Lent, séquentiel, logique.
        \item Symboles explicites, règles.
        \item Planification, audit.
        \item $\rightarrow$ \textit{Symbolic Auditor}
      \end{itemize}
    \end{column}
  \end{columns}
\end{frame}

\begin{frame}{Ancrage des Symboles (Symbol Grounding)}
  \begin{itemize}
    \item \textbf{Défi} : Comment lier le symbole abstrait « Fake News » au texte réel ?
    \item \textbf{Approche sysCRED} : Vecteurs d'embedding des LLM comme pont vers l'ontologie.
    \item \textbf{Risque} : « Raccourcis de raisonnement » (Reasoning Shortcuts) \citep{Marconato2025ReasoningShortcuts}.
      \begin{itemize}
        \item \textit{Exemple} : Associer « Crédible » au style académique superficiel.
        \item \textit{Solution} : Régularisation logique stricte.
      \end{itemize}
  \end{itemize}
\end{frame}

% --- Section 4: État de l'Art ---
\section{État de l'Art (2024-2025)}

\begin{frame}{Architectures Neuro-Symboliques (NeSy)}
  \begin{enumerate}
    \item \textbf{Pipeline} : Neuronal $\rightarrow$ Symbolique (Choix pour sysCRED pour l'explicabilité).
    \item \textbf{Co-Learning} : Contraintes logiques dans la \textit{loss function} (Logic Tensor Networks).
    \item \textbf{Agentic} : LLM + Outils externes (Translate-Infer-Compile).
  \end{enumerate}
\end{frame}

\begin{frame}{GraphRAG et Zero Trust}
  \begin{itemize}
    \item \textbf{GraphRAG} (Retrieval-Augmented Generation sur Graphes) :
      \begin{itemize}
        \item Récupère des sous-graphes, pas juste du texte.
        \item Permet le raisonnement multi-sauts \citep{Wang2025GraphCheck}.
      \end{itemize}
    \item \textbf{Zero Trust AI} :
      \begin{itemize}
        \item « Never Trust, Always Verify ».
        \item Le module symbolique audite systématiquement le neuronal.
      \end{itemize}
  \end{itemize}
\end{frame}

% --- Section 5: Méthodologie ---
\section{Méthodologie}

\begin{frame}{Design Science Research (DSR)}
  Création d'un artefact (sysCRED) pour résoudre un problème et générer des connaissances \citep{Hevner2004Design, Peffers2007DSRM}.
  
  \begin{enumerate}
    \item \textbf{Cycle de Pertinence} : Besoins en explicabilité et traçabilité.
    \item \textbf{Cycle de Conception} :
      \begin{itemize}
        \item Itération 1 : Pipeline Python/Turtle (Terminé).
        \item Itération 2 : GraphRAG + Zero Trust (En cours).
        \item Itération 3 : Optimisation et IHM.
      \end{itemize}
    \item \textbf{Cycle de Rigueur} : Ancrage dans les standards (W3C, OWL) et théorie.
  \end{enumerate}
\end{frame}

% --- Section 6: Architecture ---
\section{Architecture sysCRED}

\begin{frame}{Vue d'ensemble : Le « Sandwich Cognitif »}
  Architecture micro-services conteneurisée.
  
  \begin{enumerate}
    \item \textbf{Perception (S1)} : LLM fine-tunés (NER, Extraction Relations). Émet des assertions probabilistes.
    \item \textbf{Le Pont (Bridge)} : Traduction Vecteur $\leftrightarrow$ Symbole (Grounding).
    \item \textbf{Connaissances (Graphe)} : Neo4j + RDFLib. Mémoire à long terme.
    \item \textbf{Raisonnement (S2)} : Moteurs logiques (HermiT, Pellet). Règles SWRL.
      \begin{itemize}
        \item \textit{Règle Exemple} : Source satirique $\rightarrow$ Information fausse.
      \end{itemize}
  \end{enumerate}
\end{frame}

\begin{frame}{Flux de Traitement (Workflow)}
  \begin{enumerate}
    \item \textbf{Ingestion} : Texte/URL.
    \item \textbf{Extraction Neuronale} : Proposition de sous-graphe temporaire.
    \item \textbf{Ancrage et GraphRAG} : Contextualisation via le Knowledge Graph global.
    \item \textbf{Audit Symbolique} : Vérification de cohérence logique (Détection de contradictions).
    \item \textbf{Synthèse} : Score de crédibilité + Explication causale en langage naturel.
  \end{enumerate}
\end{frame}

% --- Section 7: Plan de Recherche ---
\section{Plan de Recherche}

\begin{frame}{Feuille de Route (2026)}
  \begin{itemize}
    \item \textbf{Phase 1 : Consolidation (Fév - Mars)}
      \begin{itemize}
        \item Finalisation ontologie (Rhétorique, Biais).
        \item Pipeline GraphRAG (LLM $\leftrightarrow$ Neo4j).
      \end{itemize}
    \item \textbf{Phase 2 : Évaluation (Mars - Avril)}
      \begin{itemize}
        \item Tests sur dataset LIAR.
        \item Étude d'ablation (avec/sans moteur de règles).
        \item Robustesse (Adversarial attacks).
      \end{itemize}
    \item \textbf{Phase 3 : Finalisation (Avril)}
      \begin{itemize}
        \item Rédaction thèse et Soutenance (3 avril 2026).
        \item Publication (ISWC, AAAI).
      \end{itemize}
  \end{itemize}
\end{frame}

% --- Conclusion ---
\section{Conclusion}

\begin{frame}{Conclusion}
  \begin{itemize}
    \item Réponse au \textit{Léviathan Algorithmique} par une approche hybride rigoureuse.
    \item \textbf{sysCRED} : L'intuition probabiliste soumise à la vérification logique.
    \item Validité théorique (DSR) et pertinence sociétale (Désinformation).
    \item Vers une IA qui rend compte de ses raisonnements.
  \end{itemize}
\end{frame}

% --- Bibliographie ---
\begin{frame}[allowframebreaks]{Références}
  \bibliography{references}
\end{frame}

\end{document}
